% !TeX root = ../main.tex

\chapter{总结}

本项目按照需求分析,项目设计,编码实践,优化程序,软件测试的步骤进行。开发者倾注了三个月的心血,终于在2021年底成功完成了本项目。

本项目的开发是一次挑战,也是一次锻炼自身能力的机会。开发本项目的经历增强了开发者的:

\begin{itemize}

\item \textbf{代码编写能力}

Unity 项目脚本需要由面向对象的 C\# 语言编写,在本项目中开发者自主编写了几千行代码,以实现众多系统与游戏功能。在学习至熟悉 C\# 语言的同时,也体会到了面向对象编程的相对优势。面向对象编程以真实世界为基础,引入了“对象”的概念。对象包含了以“属性”形式存在的数据,和以“方法”形式存在的代码。各对象自下而上组成了程序,就像各种物质组成了这个世界。面向对象编程支持重载,容易添加新的数据与函数,并且访问权限机制使其更加安全。

\item \textbf{艺术设计能力}

本项目的4个人物、游戏音乐、数学符号元素均为自主设计,开发者接触并初步掌握了3D建模技巧、音乐制作技巧以及blender等美工软件的使用。

\item \textbf{项目开发能力}

开发者学习并遵守了《Google 开源项目风格指南》中的大部分规范,函数、变量命名清晰,并使用各类注释增强代码的可读性。这有利于项目的开发与调试,为开发者间的合作提供了良好的环境,提高了开发速度。

\item \textbf{资源获取能力}

开发者主要通过 Unity 官方指南获取资料信息,同时也查阅了维基百科、Stack Overflow 等互联网资料以及李在贤编著的《Unity 5权威讲解》一书。

\end{itemize}
