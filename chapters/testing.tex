% !TeX root = ../main.tex

\chapter{软件测试}

软件测试 (software testing),即是在规定的条件下对程序进行操作,以发现程序错误,衡量软件质量,并对其是否能满足设计要求进行评估的过程。具体地,它通过比较实际输出与预期输出间的审核或者比较,来促进鉴定软件的正确性、完整性、安全性和质量。\cite{softwaretestwiki}

本游戏软件的测试分为四个阶段,分别是单元测试、集成测试和系统测试。

\section{单元测试}

单元测试 (unit testing) 是针对程序独立单元进行正确性检验的测试工作。程序单元由一组程序模块(软件设计的最小单位)组成,是软件的最小可测试部件。在过程化编程中,一个单元就是单个程序、函数、过程等;而对于面向对象编程,最小单元就是方法,包括基类(超类)、抽象类、或者派生类(子类)中的方法。通常来说,每修改一次程序就要进行最少一次单元测试,在编写程序的过程中前后很可能要进行多次单元测试,以证实程序达到软件规格书要求的工作目标,没有程序错误。\cite{unittestwiki}

在制作 patchTeX 的实践中,开发者主要通过白盒测试 (white-box testing)\footnote{测试应用程序的内部结构或运作,而不是测试应用程序的功能。在白盒测试时,以编程语言的角度来设计测试案例。测试者输入资料验证资料流在程序中的流动路径,并确定适当的输出,类似测试电路中的节点。这种测试在算法竞赛中很常见。尽管这种测试的方法可以发现许多的错误或问题,它可能无法检测未使用部分的规范。} 来进行单元测试,这是在代码编写时即可进行的测试。

\section{集成测试}

集成测试 (integration testing) 是对程序模块采用一次性或增值方式组装起来,对系统的接口进行正确性检验的测试工作。整合测试一般在单元测试之后、系统测试之前进行。实践表明,有时模块虽然可以单独工作,但是并不能保证组装起来也可以同时工作。\cite{inttestwiki}

开发者主要通过白盒测试进行集成测试,这主要在 Unity 编辑器的游戏播放模式下进行。

\section{系统测试}

系统测试 (system testing) 是将需测试的软件,作为整个基于计算机系统的一个元素,与计算机硬件、外设、某些支持软件、数据和人员等其他系统元素及环境结合在一起测试。系统测试的目的在于通过与系统的需求定义作比较,发现软件与系统定义不符合或与之矛盾的地方。\cite{systestwiki}

开发者通过黑盒测试 (black-box testing)\footnote{测试应用程序的功能,而不是其内部结构或运作。测试者不需具备应用程序的代码、内部结构和编程语言的专门知识。测试者只需知道什么是系统应该做的事,即当键入一个特定的输入,可得到一定的输出。测试案例是依应用系统应该做的功能,照规范、规格或要求等设计。测试者选择有效输入和无效输入来验证是否正确的输出。} 进行系统测试,主要手段是通过画状态转移表 (state-transition table),不重不漏地逐一测试每一种情况。(表 \ref{tab:testing})

\begin{table}[htbp]
\centering
\caption{系统测试举例}
\label{tab:testing}
\begin{tabular}{ |p{3.75cm}|p{3.75cm}|p{3.75cm}|  }
\hline
\multicolumn{3}{|c|}{在怪物攻击范围内的测试}\\
\hline
玩家输入&中间状态&结果输出\\
\hline
无操作或仅改变视角&播放怪物攻击动画&主角HP槽中血量减少\\
在怪物攻击范围内移动&播放主角移动动画、怪物追踪与攻击动画&主角与怪物方位改变、主角HP槽中血量减少\\
主角静止时按下f键或单击鼠标左键&播放主角攻击动画&怪物血量减少\\
主角移动时按下f键或单击鼠标左键&播放主角滑铲攻击动画&怪物血量减少\\
移动出怪物攻击范围&播放主角移动动画&怪物被暂时冻结\\
\hline
\end{tabular}
\end{table}
