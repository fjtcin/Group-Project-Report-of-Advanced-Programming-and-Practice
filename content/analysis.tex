\section{项目需求分析}

大物实验在\href{https://icourse.club/course/12716/}{评课社区}、\href{https://www.zhihu.com/question/35867101}{知乎}等网站上一直饱受争议,
有多名同学指出实验报告撰写耗时长、专业作图软件难以使用、Word 中打数学公式麻烦等问题。
鉴于此,曾经有学长在开发过一款\href{https://github.com/regymm/PhysicsExp}{大物实验数据处理工具},
这是非常好的创意。但是,这款软件入门成本太高,故了解它的人很少。

本小组开发的\href{http://home.ustc.edu.cn/~sunxulei/dawu/}{\emph{大雾实验工具}}是一款网页应用,无需安装任何软件,更不需要有编程基础,没有任何学习成本。
本工具的目标用户是中国科学技术大学大一本科生,着力于解决其撰写实验报告时最耗时的三件事情,即“绘制图像”“计算不确定度”“在电脑上书写公式”。

当然,一些高级软件也能出色地完成上述的本工具的功能,如专业绘图软件 Origin, 专业计算软件 Matlab 等。
但我们的项目不是去取代这些强大的软件,而是将它们\emph{本地化}。
这些软件功能繁多,故学习成本相对较高,但我们的软件为每一个大物实验都写了专门的处理工具,封装到只需要用户上传数据表格的程度。
相比动辄几个 \unit{\giga\byte} 的专业软件来说,我们的工具更加友好,更加便捷,更加有针对性——更加有效。